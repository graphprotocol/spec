\chapter{Introduction}
{\justifying

\begin{flushleft}
\textit{[A] mathematical problem should be difficult in order to entice us, yet not completely inaccessible, lest it mock at our efforts. It should be to us a guide post on the mazy paths to hidden truths, and ultimately a reminder of our pleasure in the successful solution.}
\end{flushleft}
\vspace{-0.2in}
\begin{flushright}
    -- David Hilbert\sidenote[][]{Taken from the ``Mathematical Problems" lecture delivered before the International Congress of Mathematicians   in 1900 \cite{hilbert1902mathematical}.}
\end{flushright}

\lipsum[1-4]
 
\begin{pagetable}  	
	    \centering 
	    \renewcommand{\arraystretch}{1.5}
		\begin{tabular}{c} 
			 Sample Table Goes here \\
			 Table Contents...
	\end{tabular}
	\renewcommand{\arraystretch}{1.0} 
    \caption[Sample Table Brief Caption]{Long caption goes here}	
    \label{table: sample}
\end{pagetable}  
 
 \begin{pagefigure}
     \centering
      \textcolor{red}{[Insert Dependency Chart]}
      \vspace*{0.1in}
      \caption[Chapter content dependency graph]{Graph for chapter dependencies. }
     \label{fig:my_label}
 \end{pagefigure}
 } 