\chapter{Bonding Curves}
{\justifying

\textit{Knowledge work is not an assembly line, and extracting value from information is an activity that’s often at odds with busyness, not supported by it.} \hfill -- Cal Newport \textcolor{red}{[cite]}

\vspace{0.25in}

\textcolor{red}{[This spec section will be updated using the Safe and Fair Continuous Organizations work, as that progresses.]}
 
Curators perform an essential role within The Graph. They operate as knowledge workers that investigate subgraphs and identify which of these are of quality and ought to be indexed by The Graph's APIs. For an elementary introduction, we refer readers to   existing guides \textcolor{red}{[cite]}. This chapter overviews bonding curves, principal protected variations, their incentives, the composition of two levels of bonding curves, and \textcolor{red}{[return and complete]}. 

\begin{sidefigure}
    \centering 
    \begin{tikzpicture}[scale=0.8] 
        \def\X{0.4}
        \def\Y{2}
        \def\O{1.16666}
        \def\slope{0.5} 
        
        \fill[PURPLE, opacity=0.7,    variable=\x, samples=10]
          (0, 0) 
          -- plot [domain=0:\X+\O,samples=10] (\x, { \slope * \x})
          -- (\X+\O, 0) 
          -- cycle;  
        
        \draw[line width=1.5, -latex] (-0.5,0) -- (4,0);% 
        \draw[] (2,-0.25) node[below] {Share Supply};
        
        \draw[line width=1.5, -latex] (0,-0.5) -- (0,3.25) node[right, xshift=0.1cm, yshift=-0.2cm] {Price};
 
        \draw[line width=1.3, latex-] ({0.5}, 0.5*\slope + 0.1) -- ({0.5},0.5*\slope + 1.1);
        
        \draw[] ({0.25}, 2.25) node[right, yshift=-0.05cm] {Shaded area is};
        
        \draw[]  ({0.25},1.75) node[right, yshift=-0.05cm] {curator capital}; 
        
        \draw[PURPLE!80!black, line width=1.3] plot [domain=0:\X+\O,samples=10] (\x, { \slope * \x});     
        
        \draw[PURPLE!80!black, line width=1.3, dashed] plot [domain=\X+\O:3.8,samples=10] (\x, { \slope * \x}); 
        
        \draw[fill=black] ({\X+\O},{\slope *(\X+\O)}) circle (0.1); 
        
        \foreach \x in {1,2,3}
        { 
            \draw[line width=1.5] (1.1*\x,0.2) -- (1.1*\x, -0.2) ;  
        }         
    \end{tikzpicture} 
    
    \vspace*{-0.1in}
    \caption[Linear Bonding Curve Diagram]{Example   linear bonding curve with a single curator. The   area of purple region is the  amount of GRT paid by   curator.}
    \label{fig: linear-bonding-curve}
\end{sidefigure}


\subsection{Bonding Curve Introduction}
Curators provide signal on a particular subgraph by transacting GRT to mint shares of a continuous organization (CO) \textcolor{red}{[cite white paper]} associated with that subgraph. That is, each subgraph’s bonding curve is unique. The price of shares is defined as a function $p(s)$ of the number of minted shares $s$ (\ie total currently existing supply), increasing as the number of shares increases.   In The Graph, these curves are linear (\eg see Figure \ref{fig: linear-bonding-curve}), \ie there is\sidenote[][]{\textcolor{red}{The implemented value as of writing is $k=0.03$.}} $k > 0$ such that
\begin{equation}
    p(s) = k \cdot s.
\end{equation}
Curators can burn there shares in return for GRT.
We emphasize curators can profit by burning shares if, after they mint shares, several more shares are minted by other curators.\sidenote[][]{Many discussions have ensued about phenomena relating to this feature.}
We refer readers to prior works (\eg with fixed reserve ratios) for more bonding curve context \textcolor{red}{[cite]}.

\newpage

\subsection{Composition of Bonding Curves} Two levels of bonding curves are used. The inner level, called the \textcolor{red}{deployment curve}, is accessible only to subgraph developers. The   outer curve is referred to as the  Graph Name Service (GNS) level curve, which is where exo-curators mint and burn shares. 
We define the composition of two bonding curves as follows. 
\begin{equation}
    p_g \triangleq (\mbox{GNS level curve})
    \quad \mbox{and} \quad
    p_d \triangleq (\mbox{Deployment level curve}).
    \label{eq: def-bonding-curve-levels}
\end{equation} 
The composed curve is...





\subsection{Principal-Protected Bonding Curves} 
Here we consider a new principal-protected bonding curve (PPBC) construction, unique to The Graph,\sidenote[][]{\textcolor{red}{PPBCs were initially proposed in GIP-0025. There they were proposed to be used at the subgraph deployment level.}} in which shares are minted proportional to reserves deposited. When shares are burned, users are guaranteed to be able to withdraw the amount they had deposited for those shares. In other words, usage of such  a PPBC incurs no principal risk. We restrict our attention to two archetypal curators.\sidenote[][]{Although some curators do not cleanly fit into one category, portions of their behaviors can be analyzed by considering these two archetypes.} Some agents curate solely for the reward of capital gains from bonding curves and curation royalties. Other agents (\eg some dApp developers)    curate    solely to get   a particular subgraph being indexed (as use of the  subgraph is relevant to the agents). We categorize these two types, based on motivation by exogenous and endogenous factors.
\begin{itemize}[label=$\diamond$]
    \item \textit{Exo-Curators} are   motivated by external effects of a subgraph being indexed. 
    \item \textit{Endo-Curators} seek royalties and capital gains directly obtained  by signaling.
\end{itemize} 
The non-fungibility of PPBC shares requires special considerations when designing any kind of secondary market, automated market maker (AMM) or pooled funds that accept such shares. 

\newpage
\subsection{Signal Renting}

For further details, we refer readers to the yellow paper\sidenote[][]{\textcolor{red}{This technical paper proposes and analyzes the signal renting automated market maker used by The Graph. This is tied with GIP--0029.}} \textcolor{red}{[cite]}.

}