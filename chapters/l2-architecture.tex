\chapter[Distribution of Rewards on Arbitrum]{Arbitrum Rewards}
{\justifying

\textit{You only find out who is swimming naked when the tide goes out.} 
\newline \hspace*{10pt}\hfill -- Warren Buffett \textcolor{red}{[cite]}
 

 
\section*{Introduction}
With rising gas costs, there is great community interest in a Layer 2 (L2)  scaling solution for The Graph's protocol. As mentioned in GIP-0031, several forums discussions highlight this need, and conversations among core development teams point toward optimistic rollups, and particularly Arbitrum, as a reasonable first step in this direction. This is a significant change, however, and not risk-free. For example, this may yield uncontrolled GRT production on L2, which could be a disastrous risk.\sidenote[][]{This motivates the chapter's opening quote about risk management.}  Hence, we use an Arbitrum protocol implementation that works initially as a devnet, with no indexing rewards and therefore useful for experimental subgraphs. This is a  first step towards running the protocol with rewards on Arbitrum, assuming the devnet is a successful experiment. It would be beneficial for the mechanism for rewards distribution to be implemented from the beginning, so that governance can gradually increase rewards in Arbitrum as the community gains confidence in the L2 network.

This chapter overviews deployment of The Graph Protocol to the Arbitrum One L2  blockchain.\sidenote[][]{A more expanded version of this chapter is provided in \href{https://forum.thegraph.com/t/gip-0034-the-graph-arbitrum-devnet-with-a-new-rewards-issuance-and-distribution-mechanism/3418/2}{GIP-0034}, which was principally architected by Pablo Carranza V\'elez.} We outline how the protocol coordinates between Ethereum mainnet (L1) and  the protocol on L2. The community will gradually move to L2, starting with a ``devnet'' phase in which indexing rewards are disabled, and then slowly increasing the proportion of rewards in L2. 
%Instead of minting on the RewardsManager when allocations are closed, rewards are   pre-minted with a drip function in a new Reservoir contract. This function must be called at least once per week, and sends a configurable amount of the rewards to L2.
The scheme here builds upon a prior distribution proposal on Ethereum \cite{batog2018scalable}.

\newpage

\section*{Definitions}
Several variables are used to describe the relations for rewards herein.  

\begin{subequations}
\begin{align}
    \rho & \equiv \mbox{rewards per signal} \\
    R & \equiv \mbox{rewards} \\
    p(t) & \equiv \mbox{total GRT supply produced by accumulating rewards up to time $t$} \\
    \sigma & \equiv \mbox{signal, \ie tokens on curation contract} \\
    \omega & \equiv \mbox{tokens from indexer's stake that are allocated to a subgraph} \\
    \gamma & \equiv \mbox{rewards per allocated token} \\
    r & \equiv \mbox{issuance rate (including the $+1$)} \\
    S & \equiv \mbox{set of all subgraphs} \\
    A & \equiv \mbox{set of all allocations}
\end{align}
\end{subequations}


\section*{Reward Calculations}

Reward distribution follows the approach\sidenote[][]{\textcolor{red}{In that paper, the rewards were in discrete jumps whereas here they are continuously accruing. Need to bridge this gap formally.}} of Batog \textit{et\ al}. \cite{batog2018scalable}, but here distribution is done twice: first, treating each subgraph as a staker for the total rewards (computing rewards per signal), and, secondly, treating each allocation as a staker for the subgraph’s rewards (computing rewards per allocated token). 
The global reward function $R$ is defined to encompass  the rewards functions $R_1$ and $R_2$ for L1 and L2, respectively. That is,
\begin{equation}
    R(t) = R_1(t) + R_2(t),
    \quad \mbox{for all times $t$.}
\end{equation}
The evolution of rewards is captured via ``snapshotting'' of differences,\sidenote[][-1in]{Snapshotting is a key tool for executing efficient updates in Solidity. Rather than update continuously, we aim to update precisely when there are external changes to the system. This enables us to only keep track of a snapshot of the system at a previous time in order to update it to the present.} \ie 
\begin{equation}
    R(t) = R(t_0) + \Delta R(t;\  t_0),
    \quad \mbox{for all times $t \geq t_0$,}
\end{equation}
where $\Delta R(t;\ t_0)$ is the change in rewards distributed between time $t_0$ and a later time $t$.
Rewards accrue continuously according to the issuance rate $r > 1$, and so
\begin{equation}
    \Delta R(t;\ t_0) = p(t_0) r^{t-t_0} - p(t_0),
    \quad \mbox{for all $t_0 < t < t_1$.}
\end{equation} 
The proportion of updates to each chain are defined analogously via
\begin{subequations}
\begin{align} 
    R_1(t) & = R_1(t_0) + (1-\lambda(t_0))\Delta R(t;\ t_0), \\
    R_2(t) & = R_2(t_0) + \lambda(t_0) \Delta R(t;\ t_0),
\end{align}
\end{subequations}
where $\lambda(t_0) \in [0, 1]$ is the proportion of rewards to be distributed on L2 at time\sidenote[]{We emphasize this is only valid for $t \geq t_0$ sufficiently small to give $\lambda(t) = \lambda(t_0)$. However, changes to $\lambda$ occur rarely.} $t_0$.


If $t_\sigma$ is the last time signal was updated on any subgraph, every time we want to calculate rewards, we should calculate the new value for accumulated rewards per signal as\sidenote[][]{We are interested in states of the system immediately before and after times $t$. To clarify matters around the (possible) jump discontinuities, we adopt the convention of superscript plus/minus signs for cases were we are interested in the limits as time approaches from the left for $t^-$ and from the right for $t^+$.}
\begin{equation}
    \rho(t)
    = \rho(t_\sigma) \left[1 + \dfrac{r^{t-t_\sigma} - 1}{\sum_{i\in\mathcal{S}} \sigma_i(t_\sigma^+)} \right],
    \ \ \ \mbox{for all $ t_\sigma < t < \hat{t}_{\sigma}$.}
\end{equation}
Note $\rho(t)$ is continuous since
\begin{equation}
    \rho(t_\sigma^+) = \rho(t_\sigma^-).
\end{equation}
And note time is discrete in our setting...


Then \textcolor{red}{[clarify $+$ and $-$ superscripts]}
\begin{equation}
    \Delta p_i(t;\ t_\sigma)
    = \rho_i(t) - \rho_i(t_\sigma).
\end{equation}

Then 
\begin{equation}
    p(t_0') = p(t_0) + \Delta R(t_0';\  t_0).
\end{equation}

\section*{Drip Function}
As we mentioned above, the new L1Reservoir contract will have a a drip() function. This function works as follows:
The last time the function ran, it minted tokens for all rewards up to time  $t_1$. This was saved in contract storage, together with  (So during initialization these variables will have to be set to their appropriate values when an initialization function was called, and the appropriate amount of GRT will have to be minted as well).

This drip would then be such that, if all rewards at any particular time up to $t_1$ are computed based on the stored values for, the tokens available on the Reservoir on each layer should be sufficient to cover these rewaards.


$\sigma$ is to $\rho$ what $\omega$ is to $\gamma$...

\newpage
\section*{Rewards Distribution}

\begin{equation}
    R_i(t) = R_i(t_0) + \Delta R_i(t;\ t_0).
\end{equation}
\begin{subequations}
\begin{align}
    \Delta R(t;\ t_0) & = p(t_0)r^{t-t_0} - p(t_0) \\
    \Delta R_1(t;\ t_0) & = \left(1 - \lambda(t_0)\right) \Delta R(t;\ t_0) \\
    \Delta R_2(t;\ t_0) & = \lambda(t_0) \Delta R(t;\ t_0).
\end{align}
Here $R_i$ is stored in the reservoir for each layer.

\end{subequations}


\section*{Burning Denied/Unclaimed Rewards}

\newpage
\section*{Keeper Reward for Drip Function}

\newpage
\section*{Epoch Management}

\newpage
\section*{Minting on L1 versus L1 $+$ L2}
\textcolor{red}{[single source of failure versus many]}
}